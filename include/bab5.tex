
\chapter{KESIMPULAN DAN SARAN}

\section{Kesimpulan}
Berdasarkan hasil penerapan filter spasial linear pada FPGA Development Board dengan menggunakan 6 kernel, peneliti dapat menarik beberapa kesimpulan sebagai berikut:
\begin{enumerate}[topsep=0pt,itemsep=0pt,partopsep=0pt, parsep=0pt]
    \item Proses implementasi filter spasial linear pada video \textit{stream} dengan FPGA Development Board dilakukan dengan library OpenCV python dan library xfOpenCV Xilinx. Setiap \textit{frame} dari \textit{source} video \textit{stream} direpresentasikan sebagai citra digital kemudian dilakukan filter spasial linear, selanjutnya hasil filter ini ditampilkan secara berkesinambungan sehingga tampak seperti video.
    \item Waktu komputasi dan FPS dengan menggunakan FPGA secara umum lebih baik dibandingkan dengan menggunaan prosesor ARM. Penggunaan CPU pada FPGA sedikit lebih rendah dibandingkan penggunaan CPU pada prosesor ARM. Secara umum penggunaan \textit{memory}, \textit{shared memory}, \textit{virtual memory}, dan \textit{resident memory} pada FPGA tidak jauh berbeda dengan yang digunakan pada prosesor ARM.
\end{enumerate}


\section{Saran}
Setelah melakukan penelitian ini, peneliti dapat memberikan beberapa saran sebagai berikut:
\begin{enumerate}[topsep=0pt,itemsep=0pt,partopsep=0pt, parsep=0pt]
    \item Pada penelitian ini peneliti hanya menggunakan 6 kernel dengan ukuran 3x3 yaitu \textit{average blur}, \textit{gaussian blur}, \textit{laplacian}, \textit{sharpening}, \textit{sobel horizontal} dan \textit{sobel vertical}. Untuk kedepannya, disarankan untuk menggunakan kernel lain dengan ukuran yang lebih beragam.
    \item Pada penelitian ini peneliti hanya menggunakan metode pemrosesan citra \textit{low-level} yaitu filter spasial linear, kedepannya disarankan menggunakan metode pemrosesan citra \textit{mid-level} atau pemrosesan \textit{high-level}.
    \item Melakukan pengolahan citra digital pada jenis citra warna, tidak terbatas hanya pada citra \textit{grayscale} saja.
\end{enumerate}