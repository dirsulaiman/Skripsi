
\chapter{KESIMPULAN DAN SARAN}

\section{Kesimpulan}
Berdasarkan hasil penerapan filter spasial linear pada FPGA Development Board dengan menggunakan 6 kernel, peneliti dapat menarik beberapa kesimpulan sebagai berikut:
\begin{enumerate}[topsep=0pt,itemsep=0pt,partopsep=0pt, parsep=0pt]
    \item Proses implementasi
    \item Waktu komputasi dan FPS dengan menggunakan FPGA secara umum lebih baik dibandingkan dengan menggunaan prosesor ARM. Penggunaan CPU pada FPGA sedikit lebih rendah dibandingkan penggunaan CPU pada prosesor ARM. Secara umum penggunaan memory, shared memory, virtual memory, dan resident memory pada FPGA tidak jauh berbeda dengan yang digunakan pada prosesor ARM.
\end{enumerate}


\section{Saran}
Setelah melakukan penelitian ini, peneliti dapat memberikan beberapa saran sebagai berikut:
\begin{enumerate}[topsep=0pt,itemsep=0pt,partopsep=0pt, parsep=0pt]
    \item Pada penelitian ini peneliti hanya menggunakan 6 kernel dengan ukuran 3x3 yaitu average blur, gaussian blur, laplacian, sharpening, sobel horizontal dan sobel vertical. Untuk kedepannya, disarankan untuk menggunakan kernel lain dengan ukuran yang lebih beragam.
    \item Pada penelitian ini peneliti hanya menggunakan metode pemrosesan citra low-level yaitu filter spasial linear, kedepannya disarankan menggunakan metode pemrosesan citra mid-level atau pemrosesan high-level.
    \item Melakukan pengolahan citra digital pada jenis citra warna, tidak terbatas hanya pada citra grayscale saja.
\end{enumerate}